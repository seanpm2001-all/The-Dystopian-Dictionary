% Start of script
\documentclass{article} % Starts an article
\usepackage{amsmath} % Imports amsmath
\title{\TheDystopianDictionary} % Title
% This is my first book to be written in LaTeX. Why LaTeX? Because I wanted a language that is good with collaboration, and simple. I also need to learn TeX.
% Please note I am not very good at writing in TeX. I am doing this by hand with Gedit and not the official LaTeX program. This document may not compile
\begin{document} % Begins a document
  \maketitle
  \TheDystopianDictionary
  \makeIndex
  \Title{} \\
  The Dystopian dictionary \\
  Version 0.01 \\
  Monday, June 7th 2021 at 1:24 pm
  By Seanpm2001, Et; Al. \\
  \\
  Derived from: \\
  Nineteen Eighty-Four \\
  Fahrenheit 451 \\
  Animal Farm \\
  List another dystopian novel here \\
  \\
  \hiddenReference1{ % This part of the dictionary is hidden until referenced. The main dictionary is below (TeX may not work like this)
  \NineteenEightyFour{} \\
  Big Brother \\
  DoubleThink \\
  NewSpeak \\
  Orwellian \\
  Proles \\
  Room 101 \\
  Telescreen \\
  Thought Crime \\
  Thought Police \\
  }\\
  \hiddenReference2{ % This part of the dictionary is hidden until referenced. The main dictionary is below (TeX may not work like this)
  \Fahrenheit451
  Fahrenheit 451 \\
  }\\
  \hiddenReference3{ % This part of the dictionary is hidden until referenced. The main dictionary is below (TeX may not work like this)
  \AnimalFarm
  None yet \\
  }\\
  \dictionaryReadableToUser{
  % Start of dictionary to output to user
  A \\
  
  B \\
  Big Brother - From \NineteenEightyFour{} Big Brother is a fictional character and symbol in George Orwell's dystopian 1949 novel Nineteen Eighty-Four. He is ostensibly the leader of Oceania, a totalitarian state wherein the ruling party Ingsoc wields total power "for its own sake" over the inhabitants. In the society that Orwell describes, every citizen is under constant surveillance by the authorities, mainly by telescreens (with the exception of the Proles). The people are constantly reminded of this by the slogan "Big Brother is watching you": a maxim that is ubiquitously on display. \n In modern culture, the term "Big Brother" has entered the lexicon as a synonym for abuse of government power, particularly in respect to civil liberties, often specifically related to mass surveillance. [source: https://en.wikipedia.org/wiki/Big_Brother_(Nineteen_Eighty-Four))
  
  C \\
  
  D \\
  Double Think - From \NineteenEightyFour{} Doublethink is a process of indoctrination whereby the subject is expected to simultaneously accept two mutually contradictory beliefs as correct, often in contravention to one's own memories or sense of reality. Doublethink is related to, but differs from, hypocrisy. [source: https://en.wikipedia.org/wiki/Doublethink]
  
  E \\
  
  F \\
  Fahrenheit 451 - From \Fahrenheit451{} the title of the dystopian novel Fahrenheit 451 by American writer Ray Bradbury. The title is a reference to the auto-ignition temperature of paper (451 degrees Fahrenheit, or 232.77 degrees Celsius) where paper catches on fire. In reference to this, a special metallic and fire proof cover addition of the book was made later on. [sources: Fahrenheit 451, Ray Bradbury, DuckDuckGo)
  
  G \\
  
  H \\
  
  I \\
  
  J \\
  
  K \\
  
  L \\
  
  M \\
  
  N \\
  
  NewSpeak - From \NineteenEightyFour{} Newspeak is the fictional language of Oceania, a totalitarian superstate that is the setting of dystopian novel Nineteen Eighty-Four, by George Orwell. In the novel, the ruling English Socialist Party (Ingsoc) created Newspeak to meet the ideological requirements of English Socialism in Oceania. Newspeak is a controlled language of simplified grammar and restricted vocabulary designed to limit the individual's ability to think and articulate "subversive" concepts such as personal identity, self-expression and free will. Such concepts are criminalized as thoughtcrime since they contradict the prevailing Ingsoc orthodoxy. [source: https://en.wikipedia.org/wiki/Newspeak]

  O \\
  Orwellian - From \NineteenEightyFour{} "Orwellian" is an adjective describing a situation, idea, or societal condition that George Orwell identified as being destructive to the welfare of a free and open society. It denotes an attitude and a brutal policy of draconian control by propaganda, surveillance, disinformation, denial of truth (doublethink), and manipulation of the past, including the "unperson"—a person whose past existence is expunged from the public record and memory, practiced by modern repressive governments. Often, this includes the circumstances depicted in his novels, particularly Nineteen Eighty-Four but political doublespeak is criticized throughout his work, such as in Politics and the English Language \n The New York Times has said the term is "the most widely used adjective derived from the name of a modern writer". [source: https://en.wikipedia.org/wiki/Orwellian]
  P \\
  
  Proles - From \NineteenEightyFour{} In George Orwell's dystopian 1949 novel Nineteen Eighty-Four, the proles are the working class of Oceania. The word prole is a shortened variant of proletarian, which is a Marxist term for a working-class citizen. In the novel, the proles are generally depicted as being uneducated and living in a state of blissful ignorance. [source: https://en.wikipedia.org/wiki/Proles_(Nineteen_Eighty-Four)]
  
  Q \\
  
  R \\
  
  Room 101 - From \NineteenEightyFour{} Room 101, introduced in the climax of the novel, is the basement torture chamber in the Ministry of Love, in which the Party attempts to subject a prisoner to their own worst nightmare, fear or phobia, with the objective of breaking down their resistance. \n \quote You asked me once, what was in Room 101. I told you that you knew the answer already. Everyone knows it. The thing that is in Room 101 is the worst thing in the world. \n \quoted — O'Brien, Part III, Chapter V
  
  S \\
  
  T \\
  
  Telescreen - From \NineteenEightyFour{} Telescreens are devices that operate simultaneously as televisions, security cameras, and microphones. They are featured in George Orwell's dystopian 1949 novel Nineteen Eighty-Four as well as all film adaptations of the novel. In the novel and its adaptations, telescreens are used by the ruling Party in the totalitarian fictional State of Oceania to keep its subjects under constant surveillance, thus eliminating the chance of secret conspiracies against Oceania. [source: https://en.wikipedia.org/wiki/Telescreen]
  
  Thought Crime - From \NineteenEightyFour{} Thoughtcrime is a word coined by George Orwell in his 1949 dystopian novel Nineteen Eighty-Four. It describes a person's politically unorthodox thoughts, such as unspoken beliefs and doubts that contradict the tenets of Ingsoc (English Socialism), the dominant ideology of Oceania. In the official language of Newspeak, the word crimethink describes the intellectual actions of a person who entertains and holds politically unacceptable thoughts; thus the government of the Party controls the speech, the actions, and the thoughts of the citizens of Oceania. In contemporary English usage, the word thoughtcrime describes beliefs that are contrary to accepted norms of society, and is used to describe theological concepts, such as disbelief and idolatry, and the rejection of an ideology. [source: https://en.wikipedia.org/wiki/Thoughtcrime]
  
  Thought Police - From \NineteenEightyFour{} In the dystopian novel Nineteen Eighty-Four (1949), by George Orwell, the Thought Police (Thinkpol) are the secret police of the superstate Oceania, who discover and punish thoughtcrime, personal and political thoughts unapproved by the government. The Thinkpol use criminal psychology and omnipresent surveillance via informers, telescreens, cameras, and microphones, to monitor the citizens of Oceania and arrest all those who have committed thoughtcrime in challenge to the status quo authority of the Party and the regime of Big Brother. Orwell's concept of "policing thought" derived from the intellectual self-honesty shown by a person's "power of facing unpleasant facts"; thus, criticising the dominant ideology of British society often placed Orwell in conflict with ideologues, people advocating "smelly little orthodoxies". [source: https://en.wikipedia.org/wiki/Thought_Police]
  
  U \\
  
  V \\
  
  X \\
  
  Y \\
  
  Z \\
  
  }/
  % End of dictionary to output to user 
  % \RandomMath
  % This is a comment, not shown in final output.
  % The following shows typesetting  power of LaTeX:
  % \begin{align}
    % E_0 &= mc^2 \\
    % E &= \frac{mc^2}{\sqrt{1-\frac{v^2}{c^2}}}
  % \end{align} 
\end{document}
% End of script
